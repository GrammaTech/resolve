\documentclass[letterpaper,twocolumn,10pt]{article}
\usepackage{latexsym,amssymb,stmaryrd}
\usepackage{amsmath}
\usepackage{amsthm}
\usepackage{colortbl}
\usepackage{color} 
\usepackage{graphics}
\usepackage{graphicx}
\usepackage{tikz}
\usepackage{multicol}
\usepackage{multirow}
\usepackage{url}
\usepackage{arydshln}
\usepackage{pgfplots}
\usepackage{listings}
\usepackage{todo}
\usepackage{longtable}

\usepgfplotslibrary{external}
\tikzexternalize

\let\labelindent\relax
\usepackage{enumitem}

\usepackage{ctable}
\lstset{
  frame=Ltb,
  framerule=0pt,
  %aboveskip=1em,
  framextopmargin=0pt,
  framexbottommargin=0pt,
  framexleftmargin=0cm,
  framesep=0pt,
  rulesep=0pt,
  % 
  stringstyle=\small\ttfamily,
  keepspaces=true,
  basicstyle=\fontencoding{T1}\small\fontfamily{cmtt}\selectfont\upshape,
  %basicstyle=\small\sffamily\upshape,
  showstringspaces = false,
  mathescape=true,
  breaklines=true,
  columns=fullflexible,
  %language=Prolog
}

\begin{document}

\title{Automated Merge-conflict Resolution with Semantic Differencing}

\author{
  {\rm Paul Dietz}\\
  GrammaTech, Inc.\\
  {\tt pdietz@grammatech.com}
  \and
  {\rm Jason Ruchti}\\
  GrammaTech, Inc.\\
  {\tt jruchti@grammatech.com}
  \and
  {\rm Eric Schulte}\\
  GrammaTech, Inc.\\
  {\tt eschulte@grammatech.com}
}

\maketitle

\begin{abstract}
Resolve implements multi-lingual semantic differencing of software
source code and implements an automated technique of merge conflict
resolution on top of merge conflicts over the resulting differences.
\end{abstract}

\input{intro}
\input{related}
\input{methodology}
\input{experiments}
\input{conclusion}
\input{acknowledgments}

\bibliographystyle{plain}
\bibliography{bibliography}

\end{document}
