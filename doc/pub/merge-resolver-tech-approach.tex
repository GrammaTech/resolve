\mr works in two phases.  First it leverages GrammaTech's
\gls{ast}-based software differencing (\autoref{ast-diff}) to find
semantically meaningful differences between two versions of software
against a shared base version, and to try to merge these differences
resulting in a merged piece of software with possible areas of
conflict.  Second \mr iterative steps through the identified chunks
attempting to resolve them using standard resolution schemes
identified in large-scale empirical reviews of software merging by
real developers.  The final resolution is presented to users in the
standard GitHub user interface (\autoref{github-api}).

More detail on the \mr technique is available through the web-page for
the GitHub application we built for
\mr.\footnote{\url{https://mergeresolver.github.io/}}

\begin{figure}
  \begin{center}
    \adjustbox{max width=\textwidth}{\includegraphics{figures/Screen_Shot_2020-06-11_at_10.png}}
  \end{center}
  \caption[AST-based software differencing]{\label{ast-diff}Results of
    the application of \gls{ast}-based software differencing to a
    real-world merge conflict form the popular JQuery JavaScript
    repository.}
\end{figure}

\begin{figure}
  \begin{center}
    \adjustbox{max width=\textwidth}{\includegraphics{figures/screenshot-3.png}}
  \end{center}
  \caption[\mr GitHub API]{\label{github-api}\mr GitHub API in which developers can
    accept a merge resolution.}  
\end{figure}
